\usepackage{appendixnumberbeamer}
\usepackage{graphicx}
\usepackage{url}
\usetheme[progressbar=frametitle, block=fill, numbering=fraction, sectionpage=progressbar]{metropolis}
\newcommand{\themename}{\textbf{\textsc{metropolis}}\xspace}
\usepackage{xcolor}
% \usepackage{algorithm2e}

\definecolor{UBCblue}{RGB}{106, 52, 107} % UBC Blue (primary)
\definecolor{UBCgrey}{rgb}{0.3686, 0.5255, 0.6235} % UBC Grey (secondary)

\setbeamercolor{palette primary}{bg=UBCblue,fg=white}
\setbeamercolor{palette secondary}{bg=UBCblue,fg=white}
\setbeamercolor{palette tertiary}{bg=UBCblue,fg=white}
\setbeamercolor{palette quaternary}{bg=UBCblue,fg=white}
\setbeamercolor{structure}{fg=UBCblue} % itemize, enumerate, etc
\setbeamercolor{section in toc}{fg=UBCblue} % TOC sections

\setbeamercolor{progress bar}{bg=gray,fg=UBCblue} 
\setbeamercolor{title separator}{bg=gray,fg=UBCblue}
\setbeamercolor{progress bar in head/foot}{bg=gray,fg=UBCblue} 
\setbeamercolor{progress bar in section page}{bg=gray,fg=orange}
\setbeamercolor{background canvas}{bg=white}

% Override palette coloring with secondary
\setbeamercolor{subsection in head/foot}{bg=UBCgrey,fg=white}

\usepackage{fancyhdr}
\usepackage{fancyvrb}
\usepackage{float}
\usepackage{booktabs}
\usepackage{lscape}
\usepackage{amsmath}
\usepackage{graphicx}
\usepackage{algorithm}
\usepackage{algpseudocode}
\usepackage{kpfonts}
\usepackage{tcolorbox}
\usepackage{tikz}
\DeclareMathOperator*{\argmin}{arg\,min}
\DeclareMathOperator*{\argmax}{arg\,max}
\newcommand{\theHtable}{\thetable}